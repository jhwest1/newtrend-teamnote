$$dp[i] = min_{j<i} (dp[j] + cost(j, i))$$

위와 같은 형태의 점화식에서 $cost(j, i)$가 monge 할 때 $dp[N]$를 구하고, 실제 최적해를 역추적한다.

$cost(j, i)$가 monge 하니, $p \le q \le x_1 \le x_2$ 에 대해 $cost(p, x_1)-cost(q, x_1) \le cost(p, x_2)-cost(q, x_2)$.  
$f_p(x) = dp[p] + cost(p, x)$라 하면 $f_p(x_1)-f_q(x_1) \le f_p(x_2)-f_q(x_2)$.

$f_p(x)-f_q(x)$는 증가함수이다. $(f_p(x)<f_q(x) \ \rightarrow \ f_p(x)=f_q(x) \ \rightarrow \ f_p(x)>f_q(x))$  
$x$가 증가함에 따라 $f_p(x)$가 최적이었다가, $f_q(x)$가 최적으로 바뀐다.  
$f_p(x)$와 $f_q(x)$는 대소관계가 정확히 한 번 바뀌니, 직선과 같이 취급할 수 있다.

함수들을 CHT와 비슷한 방법으로 관리한다.
두 함수 $f_p(x)$, $f_q(x)$에 대하여 $cross(p, q)$를 $f_p(x)<f_q(x)$인 최대 $x$로 정의한다.
$cross(p, q)$는 이분탐색을 통하여 구할 수 있다.

최적의 함수 $f_p(x) = dp[p] + cost(p, x)$들을 deque로 관리한다.
`cross(DQ[i-1], DQ[i]) < cross(DQ[i], DQ[i+1])`를 강제한다.
새로운 함수를 추가할 때, 불변식을 만족시키지 않는 직선들을 deque의 위쪽에서 하나씩 제거한 후, 새로운 함수를 스택에 추가한다.
DP값을 구할 때, 한 번 최적해가 아닌 함수는 앞으로도 최적해가 될 수 없음을 이용하여 최적이 아닌 함수들을 앞에서부터 하나씩 제거한 후, 최적해를 구한다.